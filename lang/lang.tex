\documentclass{article}
\usepackage{amsmath, amssymb, amsthm}
\usepackage{proof}
\usepackage{stmaryrd}
\usepackage{algorithm, algpseudocode}

\title{language for protocol}
	
\begin{document}
\maketitle

\section{A language for actors}

Expressions:
\[ e ::= n \mid b \mid e \oplus e. \]

Types for expressions will be required to be finite:
\[ \tau ::= \mathbb{Z}_{q} \mid \textsf{bool} \mid \tau \times \tau \mid \dots. \]

Let $\mathcal{I} = \{i_1, i_2, \dots \}$ be a set of \emph{interface labels} with associated types $\tau_i$. Let $\textit{Dist}$ and $\textit{Var}$ be supplies of labels for distributions and variables. Each distribution $D \in \textit{Dist}$ is assigned a type $\tau_D$ of the form $\tau_1 \to \dots \to \tau_k \to \mathcal{D}(\tau)$. (Function and distribution types do not appear elsewhere in the language.) Messages are pairs of values and interface labels.


Syntax for actors:

\begin{alignat*}{2}
    \text{decl} &::=\ && \epsilon \mid \text{handler};\ \text{decl} \\
    \text{handler} &::=\ &&\textsf{onInput}\ i\ v\ c\ \text{ where $i \in \mathcal{I}$, $v \in \textsf{Var}$} \\
    c &::=\ &&\textsf{if}\ e\ c\ c \\
    &\ \ \ \mid &&\textsf{send}\ e \to i \\
    &\ \ \ \mid &&x \leftarrow D\ e_1\ \dots\ e_k;\ c\ \text{   where $d \in \textit{Dist}$} \\
    &\ \ \ \mid &&x \leftarrow \textsf{get};\ c \\
    &\ \ \ \mid &&\textsf{put}\ e;\ c
\end{alignat*}

Upon activation, actors may receive exactly one message and deliver exactly one message. Sends are guarded behind receives.

\subsection{Typing}
Let $\Gamma$ be a typing environment, containing of variable assignments $x \mapsto \tau$ as well as a type for the state $\textsf{St}(c) \mapsto \tau$. The main typing relation is then $\Gamma \vdash c$, stating that all expressions in $c$ are well typed, all samplings in $c$ are of the correct arity and type, and all stateful commands in $c$ are well-typed.

Then, given a declaration $d$, write $\vdash d$ to mean that all of the commands in $d$ are well-typed, and all commands in $d$ share the same type for the state.

Given a message $m = (v, i)$, write $\vdash (v,i)$ to mean $\vdash v : \tau_i$.

\subsection{Interface typing}
We have an typing relation $\vdash d : I\ O$ on declarations, where $I$ and $O$ are subsets of $\mathcal{I}$.

\begin{align*}
    &\infer{\vdash \epsilon : \varnothing\ \varnothing}{} & 
       \infer{\vdash\ \textsf{onInput}\ i\ v\ c;\ d : I \cup \{i\}\ O \cup O'}{\vdash d : I\ O & \vdash c : O' & I \cap (O \cup O') = \varnothing} 
\end{align*}

Above $\vdash c : O$ is the typing relation defined by
\[\infer{\vdash \textsf{send}\ e \to i: \{i\}}{}\]
and the appropriate propogation rules.

Given a declaration $d$, let $\textsf{In}(d), \textsf{Out}(d)$ be the set of input and output messages to $d$ (i.e., elements of \textit{Msg} whose interface labels agree with the $d$). Similarly, for commands $c$, let $\textsf{Out}(c)$ be the set of output interfaces of $c$.

\subsection{Semantics}

For each type $\tau$, we have the semantic domain $\llbracket \tau \rrbracket$.
Let $\mathcal{D}$ be the monad of finite probability distributions. 
A \emph{distribution environment} $\Phi$ is a mapping $D \in \textit{Dist} \to \llbracket \tau_D \rrbracket$.

We may then give commands a denotational semantics $\llbracket c \rrbracket : \Phi \to \textsf{St}(c) \to \mathcal{D}(\textsf{Out}(c) \times \textsf{St}(c))$. 
Then, we may lift to declarations in order to obtain the semantics $\llbracket d \rrbracket : \Phi \to \textsf{In}(d) \to \textsf{St}(d) \to \mathcal{D}(\textsf{Out}(d) \times \textsf{St}(d))$. 

\section{Systems}

Our syntax for systems is:
\[\text{S} ::= \text{decl} \mid \text{S}\  \text{S}. \]


We only need one combinator: $S_1 S_2$ runs $S_1$ and $S_2$ in parallel and, if they share interface labels accordingly, these interfaces get connected together. If the interfaces of $S_1$ and $S_2$ are disjoint, then they are simply run in parallel. (In constructive crypto, there is a separate operator for parallel composition: assuming that interface labels are not reused, this is redundant.)

Lift the $\textsf{St}$ typing assignment by declaring that $\textsf{St}(S_1 S_2) = \textsf{St}(S_1) \times \textsf{St}(S_2)$.
\paragraph{Interface typing:}

\[\infer{\vdash S_1 S_2 : (I_1 \cup I_2) \setminus (O_1 \cup O_2)\ (O_1 \cup O_2) \setminus (I_1 \cup I_2)}{\vdash S_1 : I_1\ O_1 & \vdash S_2 : I_2\ O_2}\]

Lift the assignments $\textsf{In}$ and $\textsf{Out}$ according to the above rule. Define $\textsf{Connect}(S_1, S_2) = (\textsf{In}(S_1) \cap \textsf{Out}(S_2)) \cup (\textsf{Out}(S_1) \cap \textsf{In}(S_2))$; $\textsf{Connect}(S_1, S_2)$ are the message spaces for messages internal to $S_1$ and $S_2$. If $S_1$ and $S_2$ do not have any interfaces in common, $\textsf{Connect}(S_1, S_2)$ is empty. 

Systems also implicitly come with an initialization distribution $\textsf{init}(S)$, which is a distribution $D \in \textit{Dist}$ over $\textsf{St}(S)$, which takes no arguments. This initialization distribution is lifted to compositions of systems in the obvious way.

\paragraph{System semantics:}

Systems are finally given the denotational semantics $\llbracket S \rrbracket : \Phi \to \textsf{In}(S) \to \textsf{St}(S) \to \mathcal{D}(\textsf{Out}(S) \times \textsf{St}(S))_\bot$. (We adjoin $\bot$ because systems may diverge.)
This is defined to be $\llbracket S_1 S_2 \rrbracket \phi m (s_1, s_2) := \textsf{Run}_{S_1, S_2}\ m\ (s_1, s_2)$, where $$\textsf{Run}_{S_1, S_2} : \textsf{In}(S_1 S_2) \cup \textsf{Connect}(S_1, S_2) \to \textsf{St}(S_1 S_2) \to \mathcal{D}(\textsf{Out}(S_1 S_2) \times \textsf{St}(S_1 S_2))_\bot$$ is given by:

\begin{algorithm}
\begin{algorithmic}[1]
    \If {$m \in \textsf{In}(S_1)$} 
        \State $(m', s_1') \leftarrow \llbracket s_1 \rrbracket\ \phi\ m\ s_1$
        \If {$m' \in \textsf{Out}(S_1 S_2)$} \
            \State Return $(m', (s_1', s_2))$
        \Else 
            \State ($m \in \textsf{Connect}(S_1, S_2)$) 
            \State Return $\textsf{Run}\ m'\ (s_1', s_2)$
        \EndIf
    \Else
        \State ($m \in \textsf{In}(S_2)$)
        \State (This case is symmetric)
    \EndIf
\end{algorithmic}
\end{algorithm}

The above is written monadically: line two is implicitly using the $\textsf{bind}$ operation of $\textsf{D}$. The above algorithm continues to deliver the current message until an external interface is reached.


\paragraph{Equivalence of systems}

Systems $S$ and $T$ are represented by their transition functions $\delta_S = \textsf{St}(S) \to \textsf{In} \to \mathcal{D} (\textsf{Out} \times \textsf{St}(S))$, and similarly for $T$. These transition functions are coalgebras over $\mathcal{F}(X) = \textsf{In} \to \mathcal{D} (\textsf{Out} \times X$. We first form the coproduct of these two coalgebras, which is a transition function over state space $\textsf{St}(S) + \textsf{St}(T)$ which takes $S$-states to $S$-distributions, and similarly for $T$.

Now, we ask whether two initial states are bisimilar. 
Let $S$ be the state space for a $\mathcal{F}$-coalgebra. Then, for a state $s$, write $s \xrightarrow{m} \mu$ to mean $\mu := \delta_S (s,m)$.
Given an $\mathcal{F}$-coalgebra over state space $S$, we say that an equivalence relation $\sim$ is a \emph{bisimulation} if for any $s \sim t$,
\[ \forall m, \forall C \in S / \sim, \text{ if } s \xrightarrow{m} \mu \text{ and } t \xrightarrow{m} \eta \text{ then }  \forall m', \mu(m',C) = \eta(m',C),\]

where $\mu(m',C)$ is the probability that $\mu$ outputs a pair $m'', s'$ such that $m'' = m'$ and $s' \in C$.

\end{document}
