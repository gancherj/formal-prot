
\documentclass{article}
\usepackage{amsmath, amssymb, amsthm}
\usepackage{proof}
\usepackage{stmaryrd}
\usepackage{algorithm, algpseudocode}

\newcommand{\Val}{\mathsf{Val}}
\newcommand{\St}{\mathsf{St}}
\newcommand{\InBuf}{\mathsf{InBuf}}
\newcommand{\OutBuf}{\mathsf{OutBuf}}
\newcommand{\PID}{\mathsf{PID}}
\newcommand{\Var}{\mathsf{Var}}
\newcommand{\Handler}{\mathsf{Handler}}
\renewcommand{\list}{\mathsf{list}}
\newcommand{\D}{\mathcal{D}}
\newcommand{\NewEv}{\mathsf{NewEv}}
\newcommand{\OldEv}{\mathsf{OldEv}}

\title {Probabilistic Semantic Noninterference}
\begin{document}
\maketitle

\section{Parties and Schedulers}

Assume an expression language with values in $\Val$. A \emph{state} $\St$ is a map $\Var \to \Val$, where $\Var$ is a set of variable names. 
A \emph{buffer} is a value of type $\list (\PID \times \Val)$, where $\PID$ is a set of party names.
Then, a \emph{handler} $P$ is a semantic function of type $\St \to \InBuf \to \D(\St \times \OutBuf)$, where $\D$ is the finitely supported distribution monad. It is the intention that $\InBuf$ only contains the set of unprocessed input messages, and $\OutBuf$ only contains the set of output messages produced during the current invocation of $P$. It is also the intention that variables in $\St$ should not be overwritten, so that the state grows monotonically.

(Later, $P$ can be given monadically, in which we may statically track what variables are being accessed and so on.)

A \emph{scheduler} is defined by the following syntax:
\[ c := \textsf{Run}\ k\ @\ i;\ c \mid \textsf{stop}, \]

where $k \in \Handler$ is a handler name, and $i$ is a $\PID$.

\section{Semantics}

A \emph{trace} $\tau$ is a value of type $(\PID \to \St) \times \NewEv \times \OldEv$, where $\NewEv$ and $\OldEv$ are logs of the form $\list (\PID \times \PID \times \Val)$. The first component is the global state, the second component is ordered buffer of unprocessed messages, and the third component is the ordered buffer of processed messages. Given a buffer $B$, define $B_{| i}$ to be the pairs in $B$ such that the second component is equal to $i$ (preserving order).

Then, define our scheduler semantics $\llbracket c \rrbracket : \tau \to (\Handler \to P) \to \D(\tau)$ by

\[\llbracket \textsf{Run}\ k\ @\ i; c \rrbracket (G, B_u, B_p) \mathcal{I} := \textsf{bind}\ ((\mathcal{I} k) (G i) B_{u_{| i}})\ (\lambda s' o. \llbracket c \rrbracket\ (G[i := s'], o\ ||\ (B_u \setminus B_{u_{| i}}), B_{u_{| i}}\ ||\ B_p)\ \mathcal{I})\]

and

\[\llbracket \textsf{stop} \rrbracket\ \tau\ \mathcal{I} := \mathbf{1}_{\tau},\]

where $\textsf{bind}$ is the monadic bind operation for distributions and $||$ is list concatenation.

Above, $\mathcal{I}$ is an interpretation function from handler names to handlers.


(Corruption and message interference may be modeled by differing semantics.)

\section{Noninterference}

A \emph{leakage} (or \emph{declassification}) property $\varphi$ is a function $\PID \to (\PID \to \St) \to \Val$. Given an initial global state $G$, $\varphi i G$ denotes the information $i$ should be able to learn from $G$ after the execution of the protocol.

Given two distributions $D$ on traces, write $D \equiv_i D'$ if the marginals $\mathcal{D} (\lambda\ G\ B_u\ B_p.\ (G\ i, B_{p_{| i}}))$ are identical for both $D$ and $D'$. That is, $D \equiv_i D'$ if from party $i$'s position, $D$ contains exactly the same information as $D'$ on both states and processed messages.

Then, say that the pair $(c, \mathcal{I})$ is \emph{$\varphi$-noninterferent} if for all $G, G', i$, 
\[ (G\ i) = (G'\ i) \wedge \varphi\ i\ G = \varphi\ i\ G' \implies \llbracket c \rrbracket\ (G, \emptyset, \emptyset)\ \mathcal{I} \equiv_i \llbracket c \rrbracket\ (G', \emptyset, \emptyset)\ \mathcal{I}.\]

That is, $(c, \mathcal{I})$ is $\varphi$-noninterferent exactly when, for every party $i$, if two initial global states look identical to $i$ and agree on values of $\phi$, then their induced traces will appear identical to $i$.

Note that in the above definition, equivalence of traces is sensitive to order of message delivery -- but is only sensitive to message ordering from the perspective of individual parties. That is, two send commands in a handler may be safely reordered if they are sent to different recipients.

\section{Examples}

In \emph{multiparty computation}, each party is given an input $x_i$, and a protocol is devised so that each party receives the value $f(\vec{x})$, but no further information is shared. This is modeled by the leakage function $\phi\ i\ G := f((G\ 1).in, \dots, (G\ n).in)$.

Functions may also be asymmetric, in which the leakage function is $\phi\ i\ G := f_i((G\ 1).in, \dots, (G\ n).in)$. A canonical example is \emph{oblivious transfer}, where the sender has two messages $m_0$ and $m_1$, and the receiver has a bit $b$. The sender should learn nothing (i.e., $f_S (m_0, m_1, b) = ()$), while the receiver should learn the $b$th message (i.e., $f_R (m_0, m_1, b) := \text{if}\ b \text{ then } m_0 \text{ else } m_1.$)


\section{Approximate reasoning and corruption}

If we require that $D \approx_i D'$ instead of $D \equiv_i D'$, where we may bound the computational distance between distributions, then we may obtain a computational semantics for protocols. Here, $\approx_i$ means that the distributions in quesion are bounded by a negligible function of $\eta$ in distance. 

Given semantics for corrupt adversaries, a \emph{corruption model} $\mathcal{A}$ may be defined to be a set of rewrite rules on schedulers $c$. This may induce a corruption semantics $\mathcal{A}, c \vdash c'$, where $c$ is an initial, uncorrupted schedule and $c'$ is a rewriting of $c$ according to $\mathcal{A}$. Then, we may change the above main definition to universally quantify over all $c'$ such that $\mathcal{A}, c \vdash c'$.

\end{document}
