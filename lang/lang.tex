
\documentclass{article}
\usepackage{amsmath, amssymb, amsthm}
\usepackage{proof}
\usepackage{stmaryrd}
\usepackage{algorithm, algpseudocode}


\newtheorem{lemma}{Lemma}
\newcommand{\Val}{\mathsf{Val}}
\newcommand{\St}{\mathsf{St}}
\newcommand{\InBuf}{\mathsf{InBuf}}
\newcommand{\OutBuf}{\mathsf{OutBuf}}
\newcommand{\PID}{\mathsf{PID}}
\newcommand{\Var}{\mathsf{Var}}
\newcommand{\Handler}{\mathsf{Handler}}
\newcommand{\List}{\mathsf{list}}
\newcommand{\D}{\mathcal{D}}
\newcommand{\NewEv}{\mathsf{NewEv}}
\newcommand{\OldEv}{\mathsf{OldEv}}
\newcommand{\Read}{\mathsf{read}}
\newcommand{\Write}{\mathsf{write}}
\newcommand{\send}{\mathsf{send}}
\newcommand{\recv}{\mathsf{recv}}



\title {Probabilistic Semantic Noninterference}
\begin{document}
\maketitle

In this note, I give two semantics for protocols. Section 1 outlines a general message passing system, where parties are allowed to send arbitrary messages at any time. Section 2 outlines a \emph{session-restricted} system, where parties are only allowed to send messages specified by the schedule. 
The type of system in Section 2 is much more amenable to analysis; it should be the case that there exists a compiler which turns a noninterferent system in Section 2 to one in Section 1.

\section{General Message Passing}
\subsection{Parties and Schedulers}

Assume an expression language with values in $\Val$. A \emph{state} $\St$ is a map $\Var \to (\Val + \bot)$, where $\Var$ is a set of variable names. A \emph{buffer} is a value of type $\List (\PID \times \Val)$, where $\PID$ is a set of party names.

A \emph{handler} is given by the following syntax:
\[ P := x \leftarrow \Read\ \ell \mid \Write\ \ell\ e \mid \send\ i\ e \mid x \leftarrow \recv\ i \mid x \leftarrow D\ e \mid \textsf{if } e \textsf{ then } P \textsf{ else } P \mid P; P, \]

where $e$ is an expression, $\ell$ is in $\Var$, $i$ is a PID, and $D$ is a set of distributions.

Handlers are given monadic semantics as functions $\St \to \InBuf \to \D(\St \times \OutBuf)$, where $\InBuf$ and $\OutBuf$ are both buffers. Messages in the input buffer are pairs $(i,m)$, where $m$ is the message content and $i$ is the PID the message came from. Messages in the output buffer, dually, are pairs $(i,m)$, where $i$ is the PID the message is meant for. This semantics is immediate, except for $\recv$, which returns the list of all messages from $i$ in the input buffer (in reverse chronological order).

A \emph{scheduler} is defined by the following syntax:
\[ c := \textsf{Run}\ P\ @\ i \mid c_1;\ c_2, \]

where $k \in \Handler$ is a handler name, and $i$ is a $\PID$.

\subsection{Semantics}

A \emph{trace} $\tau$ is a value of type $(\PID \to \St) \times \NewEv \times \OldEv$, where $\NewEv$ and $\OldEv$ are logs of the form $\List (\PID \times \PID \times \Val)$. The first component is the global state, the second component is ordered buffer of unprocessed messages, and the third component is the ordered buffer of processed messages. Given a buffer $B$, define $B_{| i}$ to be the pairs in $B$ such that the second component is equal to $i$ (preserving order).

Then, define our scheduler semantics $\llbracket c \rrbracket : \textsf{Trace} \to \D(\textsf{Trace})$ by

\[\llbracket \textsf{Run}\ P\ @\ i \rrbracket (G, B_u, B_p) := \textsf{bind}_{\D}\ (P\ (G\ i)\ B_{u_{| i}})\ (\lambda (s', o).\ \textsf{return}_{\D} (G[i := s'], o\ ||\ (B_u \setminus B_{u_{| i}}), B_{u_{| i}}\ ||\ B_p)\ \]

and

\[\llbracket c_1;\ c_2 \rrbracket\ \tau\ := \textsf{bind}_{\D}\ (\llbracket c_1 \rrbracket \tau) \llbracket c_2 \rrbracket.\]

where $\textsf{bind}_{\D}$ and $\textsf{return}_{\D}$ are the monadic bind and return operations for distributions and $||$ is list concatenation.

\subsection{Corruption}

Extend the syntax of schedulers as so:
\[ c := \dots \mid \textsf{Corrupt}\ P\ @\ i.\]

The semantics of the added command is exactly the same as that of \textsf{Run}.

We define (static) corruption by the following rewrite rules, given by the judgement $c \rightsquigarrow_{\mathcal{A}} c'$:

\begin{tabular}{cccc}
    $\infer{c \rightsquigarrow_{\mathcal{A}} c}{}$ & 
    $\infer{\textsf{Run}\ P\ @\ i \rightsquigarrow_{\mathcal{A}} \textsf{Corrupt}\ Q\ @\ i}{}$ &
    $\infer{c_1;c_2 \rightsquigarrow_{\mathcal{A}} c_1'; c_2'} {c_1 \rightsquigarrow_\mathcal{A} c_1' & c_2 \rightsquigarrow_\mathcal{A} c_2'}$ &
    $\infer{c \rightsquigarrow_{\mathcal{A}} c'; \textsf{Corrupt}\ Q\ @\ i}{c \rightsquigarrow_\mathcal{A} c'}$ \\
\end{tabular}

Then, define $\mathsf{crupt}(c)$ to be the set of parties $i \in \PID$ such that $\textsf{Corrupt}\ Q\ @\ i$ appears in $c$ for some $Q$.

(Note that adversarial programs do not share local state. However, they may freely pass messages to and from each other.)

\subsubsection{Other adversarial models}

Above, $\rightsquigarrow_\mathcal{A}$ models full malicious corruption. We may recover semihonest corruption (i.e., ordinary party-level noninterference without byzantine faults) by replacing $\rightsquigarrow_\mathcal{A}$ with the weakened rewrite rule:

\begin{tabular}{ccc}
    $\infer{c \rightsquigarrow_{\mathcal{S}} c}{}$ & 
    $\infer{\textsf{Run}\ P\ @\ i \rightsquigarrow_{\mathcal{S}} \textsf{Corrupt}\ P\ @\ i}{}$ &
    $\infer{c_1;c_2 \rightsquigarrow_{\mathcal{S}} c_1'; c_2'} {c_1 \rightsquigarrow_\mathcal{S} c_1' & c_2 \rightsquigarrow_\mathcal{A} c_2'}$ \\
\end{tabular}

In the above rule, we do not change the semantics of any party, but only mark certain parties for corruption. By marking only a single party for corruption, the above definition collapses to ordinary noninterference.

Additionally, we may restrict our corruption model with one that cannot corrupt any party at will, but only a certain subset of the parties. An example of this is \emph{honest-verifier zero knowledge}, where the prover is permitted to be malicious but the verifier is not.

\subsection{Noninterference}

A \emph{leakage} (or \emph{declassification}) property $\varphi$ is a function $\PID \to (\PID \to \St) \to \Val$. Given an initial global state $G$, $\varphi\ i\ G$ denotes the information $i$ should be able to learn from $G$ after the execution of the protocol. Given a set $T$ of PIDs, define $\varphi\ T\ G := \{(i, \varphi\ i\ G) \mid i \in T\}$.

Given two distributions $D$ on traces, write $D \equiv_i D'$ if the marginals $\mathcal{D} (\lambda\ G\ B_u\ B_p.\ (G\ i, B_{p_{| i}}))$ are identical for both $D$ and $D'$. That is, $D \equiv_i D'$ if from party $i$'s position, $D$ contains exactly the same information as $D'$ on both states and processed messages (including order of messages). Similarly lift up to sets of parties by defining $D \equiv_T D' := \wedge_{i \in T} D \equiv_i D'$.


Fix a corruption model $\rightsquigarrow$. Given global states $G$ and $G'$, define $G =_T G'$ to be $\forall i \in T, (G\ i) = (G'\ i).$ Then, say that $c$ is \emph{$\varphi$-noninterferent} if for all $c'$ such that $c \rightsquigarrow c'$ and global states $G, G'$, 
\[ G=_{\mathsf{crupt}(c')}G' \wedge\ \varphi\ \mathsf{crupt}(c')\ G = \varphi\ \mathsf{crupt}(c')\ G' \implies \llbracket c' \rrbracket\ (G, \emptyset, \emptyset) \equiv_{\mathsf{crupt}(c')} \llbracket c' \rrbracket\ (G', \emptyset, \emptyset).\]

That is, $c'$ is $\varphi$-noninterferent  if whenever two initial global states look identical to the adversary and agree on values of $\varphi$, then their induced final traces will appear identical to the adversary. 

Note that in the above definition, equivalence of traces is sensitive to order of message delivery -- but is only sensitive to message ordering from the perspective of individual parties. That is, two send commands in a handler may be safely reordered if they are sent to different recipients.


\subsection{Authenticity}

Let $\theta$ be a property of traces $\PID \to ((\PID \to \St) \times \NewEv \times \OldEv) \to \{0,1\}$, and let $\phi$ be a property of memories $\PID \to (\PID \to \St) \to \{0,1\}$. Lift $\phi$ and $\theta$ to operate on sets of $\PID$s by defining $\phi\ T\ G := \wedge_{i \in T} \phi\ i\ G$, and similarly for $\theta$.

Fix a corruption model $\rightsquigarrow$. Then $c$ is \emph{$(\epsilon, \theta, \phi)$-authentic} if for all $c'$ such that $c \rightsquigarrow c'$ and $G$,

\[ \Pr_{\tau \leftarrow (\llbracket c' \rrbracket\ (G, \emptyset, \emptyset))}[\theta\ \mathsf{crupt}(c')\ \tau] > \epsilon \implies \phi\ \mathsf{crupt}(c')\ G.\]

That is, $c$ is \emph{$(\epsilon, \theta, \phi)$-authentic} if whenever the adversary triggers the event $\theta$ with probability larger than $\epsilon$, then $\phi$ must be true of the adversary's initial state.


\subsection{Examples}


In \emph{multiparty computation}, each party is given an input $x_i$, and a protocol is devised so that each party receives the value $f(\vec{x})$, but no further information is shared. This is modeled by the leakage function $\phi\ i\ G := f((G\ 1).in, \dots, (G\ n).in)$.

Functions may also be asymmetric, in which the leakage function is $\phi\ i\ G := f_i((G\ 1).in, \dots, (G\ n).in)$. A canonical example is \emph{oblivious transfer}, where the sender has two messages $m_0$ and $m_1$, and the receiver has a bit $b$. The sender should learn nothing (i.e., $f_S (m_0, m_1, b) = ()$), while the receiver should learn the $b$th message (i.e., $f_R (m_0, m_1, b) := \text{if}\ b \text{ then } m_0 \text{ else } m_1.$)


We may define \emph{zero-knowledge} proofs to be authentic relative to the predicates ``the verifier output 1'' and ``the prover has a correct witness $w$ to the NP-statement $x$'', and noninterferent relative to the leakage function ``$R(x,w) = 1$'' for the verifier, and no leakage for the prover. 


\section{Session-restricted Parties}

Here, we will consider parties and adversaries which are assumed to follow the protocol's intended session structure. 

Let $\St$ be as before. The type $\mathsf{Party}\ M\ M'$ is defined to be functions $\St \to M \to \D(\St \times M')$. Parties are given by the following syntax:

\[ P := x \leftarrow \Read\ \ell \mid \Write\ \ell\ e \mid \send\ e \mid x \leftarrow \recv\ \mid x \leftarrow D\ e \mid \textsf{if } e \textsf{ then } P \textsf{ else } P \mid P; P, \]

where $\recv$ evaluates to the current input message, and $\send$ alters the intended output message. (I.e., a second invocation of $\send$ will overwrite the first one.) Parties come with a typing relation $\vdash P : \mathsf{Party}\ M\ M'$, which says that all receives and sends respect $M$ and $M'$.


Schedulers have the same syntax as before:

\[ c := \textsf{Run}\ P\ @\ i \mid \textsf{Corrupt}\ P\ @\ i \mid c; c\]

Schedulers are well-typed when they respect the sessions of the parties:

    \begin{tabular}{cccc}
        $\infer{\vdash \textsf{Run}\ P\ @\ i : \textsf{Sched}\ M\ M'}{\vdash P : \mathsf{Party}\ M\ M'}$ &
        $\infer{\vdash \textsf{Corrupt}\ P\ @\ i : \textsf{Sched}\ M\ M'}{\vdash P : \mathsf{Party}\ M\ M'}$ &
        $\infer{\vdash c; c' : \textsf{Sched}\ M\ M'}{\vdash c : \textsf{Sched}\ M\ M'' & \vdash c' : \textsf{Sched}\ M''\ M'}$ &
        \ \\
    \end{tabular}

A scheduler is \emph{runnable} if it has type $\textsf{Sched}\ \textsf{unit}\ M$, for some $M$.

\subsection{Semantics}


As before, a global state is a map $\PID \to \St$. A \emph{trace} is a value of type $(\PID \to \St) \times \List (\PID \times \Val)$.

Commands are given as mappings from traces to distributions over traces:

\begin{align*}
    \llbracket \textsf{Run}\ P\ @\ i \rrbracket (G, m :: \tau) &:= \mathsf{bind} (P\ (G\ i)\ m) (\lambda (S, m').\ \mathsf{return}\ (G[i := S], m' :: m :: \tau)) \\
    \text{(Same for \textsf{Corrupt})} \\
    \llbracket c; c' \rrbracket (G, \tau) &:= \mathsf{bind} (\llbracket c \rrbracket (G, \tau)) \llbracket c' \rrbracket
\end{align*}

\subsection{Corruption}

Given a set of PIDs $T$, define

\begin{tabular}{ccc}
    $\infer{c \rightsquigarrow c}{}$ &
    $\infer{\textsf{Run}\ P\ @\ i \rightsquigarrow \textsf{Corrupt}\ Q\ @\ i}{\vdash P : \textsf{Party}\ M\ M' & \vdash Q : \textsf{Party}\ M\ M' & i \in T}$ &
    $\infer{c_1; c_2 \rightsquigarrow c'_1; c'_2}{c_1 \rightsquigarrow c'_1 & c_2 \rightsquigarrow c'_2}$
\end{tabular}


As before, define $\mathsf{crupt}(c)$ to be the set of parties in $c$ that are corrupted.

TODO: The above corruption model isn't quite right. We want adversaries to be able to share information; instead, there should be a type \textsf{AdvParty} which allows corruptions to share arbitrary state. (The above is fine for the two-party case, however)

\subsection{Trace properties}

Trace properties for this language are essentially the same as before: given a trace $\tau$ and a set of PIDs $T$, define $\tau_{| T}$ to be $T$'s visible part of $\tau$. Then, $c$ is $\varphi$-noninterferent if it is runnable and for all $c'$ such that $c \rightsquigarrow c'$ and global states $G$ and $G'$,

\[ G=_{\mathsf{crupt}(c')}G' \wedge\ \varphi\ \mathsf{crupt}(c')\ G = \varphi\ \mathsf{crupt}(c')\ G' \implies \llbracket c' \rrbracket\ (G, [()]) \equiv_{\mathsf{crupt}(c')} \llbracket c' \rrbracket\ (G', [()]).\]

Similarly, $c$ is \emph{$(\epsilon, \theta, \phi)$-authentic} if for all $c'$ such that $c \rightsquigarrow c'$ and $G$,
\[ \Pr_{\tau \leftarrow (\llbracket c' \rrbracket\ (G, [()]))}[\theta\ \mathsf{crupt}(c')\ \tau] > \epsilon \implies \phi\ \mathsf{crupt}(c')\ G,\]

where $\theta : \PID \to (\PID \to \St) \times \List (\PID \times \Val) \to \{0,1\}$, and $\phi : \PID \to (\PID \to \St) \to \{0,1\}$.

\subsection{Contexts and Indistinguishability}


In this section, fix some cost semantics for probabilistic programs.
Let $c[\_, \_, \dots, \_]$ denote multi-holed contexts where the hole ranges over \emph{distributions}. (These distributions all have type $\Val \to \D(\Val)$.)

A \emph{distribution family} is an $n$-tuple of distributions of type $\Val \to \D(\Val)$, parameterized by a natural number $\eta$ (called the \emph{security parameter}). Let $\vec{D}$ and $\vec{D'}$ be two distribution families. Let $D_i^\eta$ be the $i$th element of $\vec{D}^\eta$. Write $\vec{D} \approx \vec{D'}$ if for any polynomial time program $\mathsf{Dist}[\_, \_, \dots, \_] : \D(\{0,1\})$ with holes for the distributions in $\vec{D}$, there exists a negligible function $\epsilon$ such that
\[|\Pr[\mathsf{Dist}[D_1^\eta, \dots, D_n^\eta] = 1] - \Pr[\mathsf{Dist}[D'^\eta_1, \dots, D'^\eta_n] = 1]| < \epsilon(\eta).\]

If $\vec{D} \approx \vec{D'}$, then we say that $\vec{D}$ and $\vec{D'}$ are indistinguishable.
Finally, a schedule context $c[\_, \dots, \_]$ is \emph{polynomial time} if for every distribution family $\vec{D}$, if each $D_i^\eta$ may be executed in time polynomial in $\eta$, then $c[D_1^\eta, \dots, D_n^\eta]$ may be executed in time polynomial in $\eta$.

Then, we should recieve:
\begin{lemma}
    Let $c[\_, \_, \dots, \_]$ be a polynomial time schedule context. If $\vec{D} \approx \vec{D'}$, then the unary distribution family $\lambda \eta. \llbracket c[D_1^\eta, \dots, D_n^\eta] \rrbracket$ is indistinguishable from $\lambda \eta. \llbracket c[D'^\eta_1, \dots, D'^\eta_n] \rrbracket$.
\end{lemma}

Since our language for parties doesn't contain loops, we should also get that
\begin{lemma}
    Let $c[\_, \dots, \_]$ be a schedule context. Then $c$ is polynomial time.
\end{lemma}

(Distributions which appear in $c[\_, \dots, \_]$ cannot depend on the security parameter.)

\subsection{DDH Oblivious Transfer.}
It is often the case that schedules cannot immediately be written as schedule contexts. For example, the DDH assumption states that, for a family of cyclic groups $(g,G)^\eta$, (where $g$ is a generator of $G$)
\[(g^a, g^b, g^{ab} \mid a,b \xleftarrow{\$} \mathbb{Z}_{\mathsf{Order}(G^\eta)}) \approx (g^a, g^b, g^c \mid a,b,c \xleftarrow{\$} \mathbb{Z}_{\textsf{Order}(G^\eta)}).\]

(Above, call the left distribution $\mathsf{DDHReal}$, and the right distribution $\mathsf{DDHIdeal}$.)
The above distributions do not give the distinguisher access to the secrets $a,b$, or $c$. However, in a distributed setting, it will likely be the case that one party is holding the secret exponents, but the other party does not. In order to apply the above lemma, schedules must be massaged in order to abstract out the secret exponents. An example is oblivious transfer from DDH: given a sender with messages $m_0$ and $m_1$, and a receiver with bit $B$, the oblivious transfer protocol is


\begin{enumerate}
    \item If $B=0$, receiver generates $a,b,c$ uniformly and sends the tuple $(g^a, g^b, g^{ab}, g^c)$. If $B=1$, the last two messages are swapped.
    \item Sender, on input messages $(h_1, h_2, h_3, h_4)$, checks that $h_3 \neq h_4$. If they are unequal, they generate uniform secrets $r,s$, and send the tuple $(h_1^r g^s, m_0 h_3^r h_2^s, m_1 h_4^r h_2^s)$. If they are equal, the sender sends $(g, g, g)$ (or some other dummy message).
    \item On input $(z, w_0, w_1)$, receiver outputs $w_B / (z^b)$.
\end{enumerate}

(To output $e$ means to write $e$ to a dedicated output slot in one's local memory.)

In order to prove noninterference against the sender (so that they cannot learn the bit $B$), we first apply $\rightsquigarrow$ to the above protocol to obtain:
\begin{enumerate}
    \item If $B=0$, receiver generates $a,b,c$ uniformly and sends the tuple $(g^a, g^b, g^{ab}, g^c)$. If $B=1$, the last two messages are swapped.
    \item Sender, on input messages $(h_1, h_2, h_3, h_4)$, performs some arbitrary PPT computation and sends an output message of type $G^\eta \times G^\eta \times G^\eta$.
    \item On input $(z, w_0, w_1)$, receiver outputs $w_B / (z^b)$.
\end{enumerate}

We now note that, \emph{from the point of view of the sender}, the above protocol is equivalent to
\begin{enumerate}
    \item If $B=0$, receiver generates $a,b,c$ uniformly and sends the tuple $(g^a, g^b, g^{ab}, g^c)$. If $B=1$, the last two messages are swapped.
    \item Sender, on input messages $(h_1, h_2, h_3, h_4)$, performs some arbitrary PPT computation and sends an output message of type $G^\eta \times G^\eta \times G^\eta$.
    \item Receiver outputs $\bot$.
\end{enumerate}

The above step can be seen to be valid semantically, by noting that the third step of the protocol does not contribute to the sender's projection of the trace.

Now, we may write the above protocol as a schedule context $c[\textsf{DDHReal}, \textsf{Samp}]$:
\begin{enumerate}
    \item Receiver generates $(a,b,c)$ from $\textsf{DDHReal}$. Receiver also generates $d$ from $\textsf{Samp}$. If $B=0$, receiver sends $(a,b,c,d)$. If $B=1$, receiver sends $(a,b,d,c)$. 
    \item Sender, on input messages $(h_1, h_2, h_3, h_4)$, performs some arbitrary PPT computation and sends an output message of type $G^\eta \times G^\eta \times G^\eta$.
    \item Receiver outputs $\bot$.
\end{enumerate}

By the above lemma, the above schedule context is computationally close to $c[\textsf{DDHIdeal}, \textsf{Samp}]$, which is the same as:
\begin{enumerate}
    \item Receiver generates $(a,b,c,d)$ uniformly from $G^\eta$. If $B=0$, receiver outputs $(a,b,c,d)$; if $B=1$, receiver outputs $(a,b,d,c)$.
    \item Sender, on input messages $(h_1, h_2, h_3, h_4)$, performs some arbitrary PPT computation and sends an output message of type $G^\eta \times G^\eta \times G^\eta$.
    \item Receiver outputs $\bot$.
\end{enumerate}

One can now prove semantically that the above program satisfies noninterference against the sender. 


\end{document}
