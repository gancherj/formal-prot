\documentclass{article}

\usepackage{amsmath, amsthm}
\newcommand{\tr}{\mathsf{tr}}
\newcommand{\start}{\mathsf{start}}
\newcommand{\supp}{\mathsf{supp}}
\newcommand{\Exec}{\mathsf{Exec}}
\newcommand{\Frag}{\mathsf{Frag}}
\newcommand{\Tr}{\mathsf{Tr}}
\newcommand{\Task}{\mathsf{Task}}
\newcommand{\lst}{\mathsf{lstate}}
\newcommand{\fst}{\mathsf{fstate}}
\newcommand{\D}{\mathcal{D}}

\newtheorem{theorem}{Theorem}
\newtheorem{lemma}{Lemma}

\begin{document}
\title{PIOA Simulations}

\maketitle

Given a PIOA $P$, let $\Tr_P$ be the set of all valid transitions of $P$: transitions $s \xrightarrow{a} t$ such that $t \in \supp(\tr\ s\ a)$. Given a task $T$ and a state $s$ such that $T$ is enabled in $s$, write $Ts$ to be the induced distribution on $\Tr_P$. (By input enabledness, we
may also write $is$ for any input action $i$ to be the corresponding induced distribution.) Lift this to execution fragments $\alpha$ such that $T$ is enabled in $\lst \alpha$ in the obvious way. Then, lift this to distributions of execution fragments $\mu$, by sampling $\alpha$ from $\mu$, and returning $T\ \alpha$ if $T$ is enabled in $\lst \alpha$, and $\alpha$ otherwise. (Note by this definition, $T\ \mu$ is equal to $\textsf{apply}(T, \mu)$.)

Given a task sequence $\sigma$, say that $\sigma$ is disabled in $s$ if all tasks in $\sigma$ are disabled in $s$, and that $\sigma$ is enabled in $s$ if there exists a task in $\sigma$ enabled in $s$.

Given two comparable (not necessarily closed) task-structured PIOAs $P$ and $P'$, a state simulation from $P$ to $P'$ is a tuple of functions $\phi_S : S_P \to S_{P'}$, $\phi_T : \Tr_P \to \Frag_{P'}$, $C : \Task_P \to \Task_{P'}^*$ such that:
\begin{enumerate}
    \item $\phi_S\ \start_P = \start_{P'}$
    \item for all $t \in \Tr_P$, $\phi_S (\fst\ t) = \fst\ (\phi_T\ t)$
    \item for all $t$, $\phi_S (\lst\ t) = \lst\ (\phi_T\ t)$
    \item for all $t$, $\tr\ t = \tr\ (\phi_T\ t)$
    \item for all $s$ and tasks $T$, $T$ is enabled in $s$ iff $C(T)$ is enabled in $\phi_S\ s$
    \item for all tasks $T$ and states $s$ enabled in $T$, $\phi_T (Ts) = C(T)\ (\phi_S\ s)$
    \item for all input actions $i$, $\phi_t\ (i s) = i (\phi_S s)$.
\end{enumerate}

\begin{theorem}
    If there exists a state simulation from $P$ to $P'$ then $P \leq_0 P'$.
\end{theorem}

The above is proved in two parts:
\begin{lemma}[Soundness]
    If $P$ and $P'$ are closed and there exists a state simulation from $P$ to $P'$, then there exists a simulation from $P$ to $P'$.
\end{lemma}

\begin{lemma}[Compositionality]
    If there exists a state simulation from $P$ to $P'$, then for all compatible environments $E$, there exists a state simulation from $P || E$ to $P' || E$.
\end{lemma}

\section{Soundness}
Let $(\phi_S, \phi_T, C)$ be a state simulation from $P$ to $P'$. Define $\overline{\phi} : \Frag_P \to \Frag_{P'}$ such that $\overline{\phi}(s) = \phi_S\ s$, and otherwise inductively by 
\[\overline{\phi}\ \alpha = \begin{cases}
        \phi_T\ (s \xrightarrow{a} s') & \text{ if $\alpha = s \xrightarrow{a} s'$ } \\
        (\overline{\phi}\ \alpha')^\frown (\phi_T\ (\lst\ \alpha' \xrightarrow{a}\ s)) & \text{ if $\alpha = \alpha'\ \xrightarrow{a}\ s$}
    \end{cases} \]

Define the simulation relation $R \subseteq \D(\Exec\ P) \times \D(\Exec\ P')$ to be $\mu\ R\ \eta$ iff $\D(\overline{\phi})(\mu) = \eta$. Observe that $\phi_T$ preserves traces, so the trace condition for $R$ holds. Also, observe the start condition for $R$ holds, by the definition of a state simulation.

Now, we must verify that the step condition for $R$ holds. Choose the task correspondence $C$ to be that in the state simulation. Lift $C$ to operate on task sequences, by concatenating each image. Suppose that $\mu\ R\ \eta$ (i.e., that $\D(\overline{\phi})\ \mu = \eta$), $\mu$ is consistent with $\sigma$, $\eta$ is consistent with $C(\sigma)$. Then we must verify that for all $T$, $\D(\overline{\phi})(T\ \mu) = C(T)\ \eta$; that is, we must show that $\D(\overline{\phi})(T\ \mu) = C(T)\ \D(\overline{\phi})\ \mu$.

\begin{lemma}
    For all execution fragments $\alpha$ such that $T$ is enabled in $\lst\ \alpha$, $\overline{\phi}(T \alpha) = C(T) (\overline{\phi}\ \alpha)$.
\end{lemma}

The above follows from condition 6 in the definition of a state simulation.

\begin{lemma}
    For all distributions $\mu$ of execution fragments, $\D(\overline{\phi})(T\ \mu) = C(T)\ (\D(\overline{\phi}) \mu)$.
\end{lemma}
\begin{proof}
    Write $\mu = \sum_i p_i \alpha_i$. Then, 
    \[T\ \mu = \sum_i p_i \cdot \begin{cases} T\ \alpha_i & \text{ if $T$ is enabled in $\lst\ \alpha_i$} \\
                                        \alpha_i & \text{ otherwise } \end{cases} \] 
    thus
    \[\D(\overline{\phi})(T\ \mu) = \sum_i p_i \begin{cases} \overline{\phi}\ (T\ \alpha_i) & \text{ if $T$ is enabled in $\lst\ \alpha_i$} \\
            \overline{\phi}\ \alpha_i & \text{ otherwise } \end{cases} \] 

    By conditions 3, 5 and 6 in the state simulation, this is equal to
    \[\sum_i p_i \begin{cases} C(T) (\overline{\phi}\ \alpha_i)  & \text{ if $C(T)$ is enabled in $\overline{\phi}\ \alpha_i$} \\
            \overline{\phi}\ \alpha_i & \text{ otherwise } \end{cases} \] 
    which is equal to $C(T)\ (\D(\overline{\phi})\ \mu)$.

\end{proof}

\section{Compositionality}
Let $(\phi_S, \phi_T, C)$ be a state simulation from $P$ to $P'$, and let $E$ be compatible with $P$ and $P'$. Then, define the following state simulation from $P || E$ to $P' || E$:
\begin{align*}
    \phi'_S (s,t) &= (\phi_S\ s, t) \\
    \phi'_T ((s,t) \xrightarrow{a} (s', t')) &=
        \begin{cases}
            \mathsf{lift}_t^{t'} (\phi_T (s \xrightarrow{a} s')) & \text{ if $a$ is enabled in $s$ } \\
            (s,t) \xrightarrow{a} (s', t') & \text{ otherwise } 
        \end{cases} \\
    C' T &= \begin{cases}
        C(T) & \text{ if $T \in \Task_P$ } \\
        T & \text{ otherwise }
    \end{cases}
\end{align*}

where $\mathsf{lift}_t^{t'}\ \alpha$ lifts each transition in $\Tr_{P'}$ to a transition in $\Tr_{P' || E}$, such that $t$ is the initial state of $E$ and $t'$ is the final state. This is always possible, since we know that $(s,t) \xrightarrow{a} (s', t')$ and that $\phi_T$ only schedules hidden actions, except possibly action $a$ if $a$ is an external action.



\end{document}
